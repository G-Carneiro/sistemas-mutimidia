%! Author = gabriel
%! Date = 11/9/21

% Preamble
\documentclass[12pt, a4paper, oneside]{abntex2}

% Packages
\usepackage{amsmath}
\usepackage{../setup/packages}

\input{../setup/setup.tex}

% Document
\begin{document}
    \printcoverufsc

    \begin{enumerate}
        \item \textbf{O CUI.2 implementa alguma compressão na imagem bmp?
                Justifique sua resposta.}

                Não, o domínio é apenas alterado de RBG para YCbCr.

        \item \textbf{Indique o PSNR medindo a perda de qualidade das imagens obtidas a partir dos arquivosCUIF.1 (lena1.bmp) e CUIF2 (lena2.bmp) com a imagem original (lena.bmp).
        Há perdas nos dados da imagem na conversão RGB → YCbCr → RGB? Se houver perda de qualidade na conversão, explique a fonte desta perda.}

                PSNR(lena.bmp, lena1.bmp) = $\infty$ dB. \\
                PSNR(lena.bmp, lena2.bmp) $\approx$ 24.7 dB. \\
                Há perdas na conversão.
                Isso se deve ao fato de se utilizar operações de ponto flutuante nas conversões, causando pequenos erros que vão se acumulando.

        \item \textbf{Informe a taxa de compressão obtida pelo CUIF.1 e pelo CUIF.3 (que usa codificação de Huffman) para a imagem lena.bmp (razão entre o arquivo bmp e o arquivo cuif).
        Observando o histograma e a tabela de codificação de Huffman (impressa quando utilizada o comando bmp2cuif -v 3), indique o símbolo que ocorre mais e o símbolo que ocorre menos neste arquivo (visto no histograma) e a codificação de Huffman para estes símbolos (visto na tabela de codificação de Huffman).}

            lena.bmp = 196,662 B. \\
            lena1.cuif = 196,628 B. \\
            lena3.cuif = 170,413 B. \\
            compressão CUIF.1 =$\frac{196,662}{196,628} \approx 1$. \\
            compressão CUIF.3 =$\frac{196,662}{170,413} \approx 1.15$. \\
            O símbolo que mais ocorre é 161, sua codificação é 000000. \\
            O símbolo que menos ocorre é 189, sua codificação é 1111111111.

        \item \textbf{Indique o PSNR comparando a imagem original lena.bmp com a imagem obtida a partir do arquivo CUIF.3 (lena3.bmp).
        Há perdas nos dados da imagem?
        Explique porquê.}

        \item \textbf{Qual a taxa de compressão obtida pelo CUIF.4 (lena4.cuif) para a imagem lena.bmp?
        Para esta imagem, qual técnica de compressão obteve maior taxa de compressão?
        Codificação de Huffman ou RLE?}

        \item \textbf{Indique a PSNR das codificações CUIF.4 (erro do lena4.bmp em relação à lena.bmp).Compare o valor obtido com a PSNR do CUIF3 e justifique os resultados.}

        \item \textbf{Codifique as imagens lena.bmp e lena.bmp usando CUIF.4. Qual imagem obteve maior compressão?
        Explique porquê.}


    \end{enumerate}
\end{document}