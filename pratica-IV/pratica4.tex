%! Author = gabriel
%! Date = 11/9/21

% Preamble
\documentclass[12pt, a4paper, oneside]{abntex2}

% Packages
\usepackage{amsmath}
\usepackage{../setup/packages}

\input{../setup/setup.tex}

% Document
\begin{document}
    \printcoverufsc

    \begin{enumerate}
        \item O CUI.2 implementa alguma compressão na imagem bmp?
                Justifique sua resposta.

        \item Indique o PSNR medindo a perda de qualidade das imagens obtidas a partir dos arquivosCUIF.1 (lena1.bmp) e CUIF2 (lena2.bmp) com a imagem original (lena.bmp).
            Há perdas nos dados da imagem na conversão RGB → YCbCr → RGB? Se houver perda de qualidade na conversão, explique a fonte desta perda.

        \item Informe a taxa de compressão obtida pelo CUIF.1 e pelo CUIF.3 (que usa codificação deHuffman) para a imagem lena.bmp (razão entre o arquivo bmp e o arquivo cuif).
            Observando o histograma e a tabela de codificação de Huffman (impressa quando utilizada o comando bmp2 cuif -v 3), indique o símbolo que ocorre mais e o símbolo que ocorre menos neste arquivo (visto no histograma) e a codificação de Huffman para estes símbolos (visto na tabela de codificação de Huffman).

        \item Indique o PSNR comparando a imagem original lena.bmp com a imagem obtida a partir doarquivo CUIF.3 (lena3.bmp).
            Há perdas nos dados da imagem?
            Explique porquê.

        \item Qual a taxa de compressão obtida pelo CUIF.4 (lena4.cuif) para a imagem lena.bmp?
                Para esta imagem, qual técnica de compressão obteve maior taxa de compressão?
                Codificação de Huffman ou RLE?

        \item Indique a PSNR das codificações CUIF.4 (erro do lena4.bmp em relação à lena.bmp).Compare o valor obtido com a PSNR do CUIF3 e justifique os resultados.

        \item Codifique as imagens lena.bmp e lena.bmp usando CUIF.4. Qual imagem obteve maior compressão?
                Explique porquê.


    \end{enumerate}
\end{document}