%! Author = gabriel
%! Date = 11/9/21

% Preamble
\documentclass[12pt, a4paper, oneside]{abntex2}

% Packages
\usepackage{../setup/packages}

\input{../setup/setup.tex}
\renewcommand{\theenumi}{\Alph*}
\renewcommand{\theenumii}{\roman*}

% Document
\begin{document}
    \printcoverufsc

    Faça as conversões abaixo acumuladamente usando o software Audacity e responda as questões:

    \begin{enumerate}[ref=\theenumi]
        \item\label{item:A} Em relação ao arquivo, responda
            \begin{enumerate}
                \item qual é o tamanho teórico do áudio (parte de dados);
                \item observando as propriedades do arquivo em seu sistema
                    operacional (no linux execute du -s -B1 audio.wav),
                    e indique quais motivos o tamanho do arquivo em disco é maior que o tamanho teórico;
                \item qual seria o tamanho deste arquivo em disco se
                        o seu HD fosse formatado para um tamanho de bloco de 2048 bytes?
            \end{enumerate}
        \item\label{item:B} Baixe taxa de amostragem para 8000Hz (sem alteração do número de bits
            por amostra), e responda:
            \begin{enumerate}
                \item o tamanho teórico da mídia;
                \item qual a frequência do maior componente frequência teórico para o novo formato do áudio;
                \item explique o efeito que ocorreu no som e explique porque ocorreu os períodos de silêncio
                    no áudio convertido. Lembre-se de visualizar as trilhas do som no software Audacity para
                    responder a pergunta.
            \end{enumerate}
        \item\label{item:C} Após reduzir a taxa de amostragem em \ref{item:B}, reduza também o número de bits
            por amostra pela metade (8 bits por amostra). Abra o arquivo salvo e indique em seu relatório:
            \begin{enumerate}
                \item o tamanho teórico da mídia;
                \item qual a frequência do maior componente frequência teórico para o novo formato do áudio;
                \item explique o efeito na qualidade do áudio gerada pela redução do número de bits por
                    amostra (ouça os períodos de silêncio em \ref{item:B} e comparece com \ref{item:C}.
            \end{enumerate}
    \end{enumerate}
\end{document}