%! Author = gabriel
%! Date = 11/9/21

% Preamble
\documentclass[12pt, a4paper, oneside]{abntex2}

% Packages
\usepackage{amsmath}
\usepackage{../setup/packages}

%! Author = gabriel
%! Date = 5/17/21

% --------------------------------------------
% Insira o nome do(s) autor(es).
% Caso seja mais de um, insira \and entre os
% nomes de cada um.
% --------------------------------------------
\author{Gabriel Medeiros Lopes Carneiro \\
        Lorenzo Lima Franco Maturano}

% --------------------------------------------
% Insira o nome da universidade.
% --------------------------------------------
\university{Universidade Federal de Santa Catarina}

% --------------------------------------------
% Insira o nome do centro de ensino.
% --------------------------------------------
\educationcenter{Centro Tecnológico}

% --------------------------------------------
% Insira o nome do departamento de ensino.
% --------------------------------------------
\department{Departamento de Informática e Estatística}

% --------------------------------------------
% Insira o nome do curso ao qual pertence.
% --------------------------------------------
\course{Ciências da Computação}

% --------------------------------------------
% Afiliação do autor.
% -------------------------------------------- 
\affil{\printuniversity \par
        \printeducationcenter \par
        \printdepartment \par
        \printcourse}

%! Author = gabriel
%! Date = 5/17/21

% --------------------------------------------
% Insira o título do trabalho.
% --------------------------------------------
\title{Prática I}
% --------------------------------------------
% Caso o trablalho tenha subtítulo, descomente
% a linha abaixo.
% OBS.: NÃO APAGAR ":~" irá desconfigurar o arquivo.
% --------------------------------------------
%\subtitle{:~Subtítulo (se houver)}

% --------------------------------------------
% Insira o tipo de trabalho do documento.
% --------------------------------------------
\worktype{Tipo do trabalho}

% --------------------------------------------
% Insira o local de apresentação do documento.
% --------------------------------------------
\local{Florianópolis, SC}

% --------------------------------------------
% Define a data do documento. Por padrão mostra
% apenas o ano, caso queira a data completa,
% substitua por \today.
% --------------------------------------------
\date{\the\year}

% --------------------------------------------
% Caso o trabalho possua um orientador,
% comente a linha abaixo.
% --------------------------------------------
\orientador[Professor:]{Nome completo do professor}

% --------------------------------------------
% Caso o documento possua um orientador,
% descomente a linha abaixo.
% --------------------------------------------
%\orientador{Nome completo do orientador}

% --------------------------------------------
% Caso o documento possua um coorientador,
% descomente a linha abaixo.
% --------------------------------------------
%\coorientador{Nome completo do coorientador}

% --------------------------------------------
% Substituir '[mestre/doutor] em título obtido'
% pelo grau adequado.
% --------------------------------------------
%\formation{mestre/doutor em título obtido}

% --------------------------------------------
% Substituir nome do curso pelo nome do curso.
% --------------------------------------------
%\program{Programa de Pós-Graduação em nome do curso}

% --------------------------------------------
% Caso precise do preâmbulo do documento,
% descomente as linhas abaixo. Ele deve conter,
% o tipo do documento, o objetivo, o nome da
% instituição e a área de concentração.
% --------------------------------------------
%\preambulo{
%    \printworktype~ submetida ao
%    \printprogram~ da \printuniversity~
%    para a obtenção do título de \printformation.
%}

% --------------------------------------------
% Definição de cores de hyperlinks
% e formatações do pdf.
% --------------------------------------------
\hypersetup{
    colorlinks=true,
    linkcolor=black,
    filecolor=magenta,
    urlcolor=blue,
    citecolor=black,
    pdfauthor=\theauthor,
    pdftitle=\thetitle,
    bookmarksopen=true,
}
\renewcommand{\theenumi}{\Alph*}
\renewcommand{\theenumii}{\roman*}

% Document
\begin{document}
    \printcoverufsc

    Sabendo que o arquivo audio.wav possui (44.1KHz, 16 bits/amostra, mono) como características.
    Faça as conversões abaixo acumuladamente usando o software Audacity e responda as questões:

    \begin{enumerate}[ref=\theenumi]
        \item\label{item:A} Em relação ao arquivo, responda
            \begin{enumerate}[ref=\theenumi.\theenumii]
                \item qual é o tamanho teórico do áudio (parte de dados);
                    \[
                    \begin{matrix}
                        \text{tamanho\_teórico} & = & \dfrac{\text{num\_canais} \cdot \text{amostras/s} \cdot \text{bits/amostra} \cdot \text{duração}}{8} \\ \\
                        & = &   \dfrac{1 \cdot 44100 \cdot 16 \cdot 20}{8} \\ \\
                        & = &   1764000\text{B}
                    \end{matrix}
                    \]
                \item\label{subitem:ii} observando as propriedades do arquivo em seu sistema
                    operacional (no linux execute du -s -B1 audio.wav),
                    e indique quais motivos o tamanho do arquivo em disco é maior que o tamanho teórico;
                    \begin{itemize}
                        \item O arquivo não é formado somente de dados, há também metadados que ocupam espaço de disco.
                        \item A diferença também se deve ao fato de que o sistema operacional só aloca múltiplos do tamanho do bloco, se os dados não ocupam todo o bloco haverá desperdício.
                    \end{itemize}
                \item qual seria o tamanho deste arquivo em disco se o seu HD fosse formatado para um tamanho de bloco de 2048 bytes?

                    \[
                        \begin{matrix}
                            \text{tamanho} & = & \left\lceil \dfrac{\text{tamanho\_arquivo}}{tamanho\_bloco} \right\rceil \cdot \text{tamanho\_bloco}  \\ \\
                            & = & \left\lceil \dfrac{1764000}{2048} \right\rceil \cdot 2048  \\ \\
                            & = &   862 \cdot 2048 \\ \\
                            & = &   1765376\text{B}
                        \end{matrix}
                    \]
                    Esse valor é o mesmo que foi obtido no item \ref{subitem:ii}, quando foram observadas as propriedades do arquivo no sistema operacional.

            \end{enumerate}
        \item\label{item:B} Baixe taxa de amostragem para 8000Hz (sem alteração do número de bits
            por amostra), e responda:
            \begin{enumerate}
                \item o tamanho teórico da mídia;
                    \[
                        \begin{matrix}
                            \text{tamanho\_teórico} & = & \dfrac{\text{num\_canais} \cdot \text{amostras/s} \cdot \text{bits/amostra} \cdot \text{duração}}{8} \\ \\
                            & = &   \dfrac{1 \cdot 8000 \cdot 16 \cdot 20}{8} \\ \\
                            & = &   320000\text{B}
                        \end{matrix}
                    \]

                \item qual a frequência do maior componente frequência teórico para o novo formato do áudio;

                    Segundo o Teorema de Nyquist, a frequência do maior componente frequência teórico é metade da taxa de amostragem. Logo,
                    \[
                        \dfrac{8000}{2} = 4000\text{Hz}
                    \]
                \item explique o efeito que ocorreu no som e explique porque ocorreu os períodos de silêncio no áudio convertido. Lembre-se de visualizar as trilhas do som no software Audacity para
                    responder a pergunta.

                    Como a taxa de amostragem foi reduzida, a maior componente de frequência também sofreu redução, como visto no item anterior. Portanto, as frequências acima desse limite foram convertidas em momentos de silêncio nesse novo processamento do áudio.
            \end{enumerate}
        \item\label{item:C} Após reduzir a taxa de amostragem em \ref{item:B}, reduza também o número de bits por amostra pela metade (8 bits por amostra). Abra o arquivo salvo e indique em seu relatório:
            \begin{enumerate}
                \item o tamanho teórico da mídia;
                \[
                    \begin{matrix}
                        \text{tamanho\_teórico} & = & \dfrac{\text{num\_canais} \cdot \text{amostras/s} \cdot \text{bits/amostra} \cdot \text{duração}}{8} \\ \\
                        & = &   \dfrac{1 \cdot 8000 \cdot 8 \cdot 20}{8} \\ \\
                        & = &   160000\text{B}
                    \end{matrix}
                \]

                \item qual a frequência do maior componente frequência teórico para o novo formato do áudio;

                    Como a taxa de amostragem permanece igual ao processamento feito no item \ref{item:B}, a frequência do maior componente frequência continua sendo 4000Hz.

                \item explique o efeito na qualidade do áudio gerada pela redução do número de bits por
                    amostra (ouça os períodos de silêncio em \ref{item:B} e comparece com \ref{item:C}.

                    Ao reduzir o número de bits por amostra, os ruídos de quatização aumentam. Durante os momentos de silêncio esse aumento no nível de ruído fica mais claro.
            \end{enumerate}
    \end{enumerate}
\end{document}